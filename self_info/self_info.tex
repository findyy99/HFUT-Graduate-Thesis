% 模板设置

\newcommand{\privacy}[1][密级]{#1} % 密级
\newcommand{\type}[1][【设计】]{#1} % 类型
\newcommand{\titleCn}[1][A闸除险加固工程初步设计]{#1} % 中文题目
\newcommand{\titleEn}[1][\LaTeX -based HFUT Thesis Template]{#1} % 英文题目

\newcommand{\keywordsCn}[1][XXX;XXX;XXX;XXX;XXX]{#1} % 中文关键字
\newcommand{\keywordsEn}[1][×××; ×××; ×××; ×××; ×××]{#1} % 英文关键字

\newcommand{\supervisor}[1][【王艳巧】【副教授】]{#1} % 导师姓名
\newcommand{\studentID}[1][2017213031]{#1} % 学号
\newcommand{\studentNameCn}[1][王建召]{#1} % 填写中文姓名
\newcommand{\studentNameEn}[1][Wang Jianzhao]{#1} % 填写英文姓名

\newcommand{\finishedYear}[1][\the\year]{#1} % 论文完成日期: 年
\newcommand{\finishedMonth}[1][\the\month]{#1} % 论文完成日期: 月
\newcommand{\finishedDay}[1][\the\day]{#1} % 论文完成日期: 日


\newcommand{\department}[1][【土木与水利工程学院】]{#1} % 学院名称
\newcommand{\major}[1][【水利水电工程专业】]{#1} % 专业名称
\newcommand{\enrolmentYear}[1][【2017级】]{#1} % 入学年份



% 字号设置
\newcommand{\chuhao}{\fontsize{42.16pt}{\baselineskip}\selectfont}     % 初号字号设置
\newcommand{\xiaochuhao}{\fontsize{36.14pt}{\baselineskip}\selectfont} % 小初字号设置
\newcommand{\yichu}{\fontsize{32pt}{\baselineskip}\selectfont}      % !!!一初字号设置
\newcommand{\yihao}{\fontsize{26.097pt}{\baselineskip}\selectfont}      % 一号字号设置
\newcommand{\erhao}{\fontsize{22.08pt}{\baselineskip}\selectfont}      % 二号字号设置
\newcommand{\xiaoerhao}{\fontsize{18pt}{\baselineskip}\selectfont}  % 小二号字号设置
\newcommand{\sanhao}{\fontsize{15pt}{\baselineskip}\selectfont}  % 三号字号设置
\newcommand{\sihao}{\fontsize{14pt}{\baselineskip}\selectfont}      % 四号字号设置
\newcommand{\xiaosihao}{\fontsize{12pt}{\baselineskip}\selectfont}  % 小四号字号设置
\newcommand{\wuhao}{\fontsize{10.539pt}{\baselineskip}\selectfont}    % 五号字号设置
\newcommand{\xiaowuhao}{\fontsize{9pt}{\baselineskip}\selectfont}   % 小五号字号设置
\newcommand{\liuhao}{\fontsize{7.528pt}{\baselineskip}\selectfont}  % 六号字号设置
\newcommand{\qihao}{\fontsize{5.521pt}{\baselineskip}\selectfont}    % 七号字号设置

% 下划线
\newcommand{\underlineFixlen}[2][3.5cm]{\underline{\makebox[#1][c]{#2}}} %重新定下划线命令

%renewenvironment用法:{新环境名称}[参数个数][参数默认值]{开始部分定义}{结束部分定义}
% 中文摘要
\renewenvironment{abstract}{ % 定义新环境 abstract 
\thispagestyle{empty} % 去掉页码

{
\begin{center}
\Large \songti \bfseries 摘\hspace{1em}要\vspace{1.1cm} %摘要标题设置 large 
\end{center}
}

\setlength{\parindent}{2em}
\setlength{\parskip}{0em}
\setlength{\baselineskip}{22pt} % (宋体,小四;固定行距22磅,段前、段后均为0行间距。段落首行缩进2字符。)
\songti
}{
\setlength{\parindent}{0em} %首行缩进
\setlength{\parskip}{1em} %行间距
{\par \songti \bfseries{关键词:}} 
\keywordsCn
\clearpage
}

% 英文摘要
\newenvironment{abstractEn}{
\thispagestyle{empty} % 去掉页码
{
\begin{center}
\Large \bfseries ABSTRACT\vspace{1.5cm}
\end{center}
}
\setlength{\parindent}{1em}
\setlength{\parskip}{0em}
\setlength{\baselineskip}{22pt} % 22磅行距,首行缩进1字符,段前、段后均为0行间距
}{
\setlength{\parindent}{0em}
\setlength{\parskip}{1em}
{\par \bfseries{KEYWORDS:}}
\keywordsEn
\clearpage
}




% 目录名
\renewcommand\contentsname{ 		%没懂
\begin{center}
\songti \Large \bfseries 目\hspace{1em}录 % (宋体,小二号,加粗;居中,单倍行距,段前0.5行、段后1.5行间距)
\end{center}
\vspace{1em}
}
% 插图清单
\renewcommand\listfigurename{
\begin{center}
\songti \Large \bfseries 插图清单 % (宋体,小二号,加粗;居中,单倍行距,段前0.5行、段后1.5行间距)
\end{center}
\vspace{1em}
}

% 表格清单
\renewcommand\listtablename{
\begin{center}
\songti \Large \bfseries 表格清单 % (宋体,小二号,加粗;居中,单倍行距,段前0.5行、段后1.5行间距)
\end{center}
\vspace{1em}
}

\renewcommand\refname{\heiti \sanhao \bfseries 参考文献}


% 目录引线设置
\renewcommand{\cftdotsep}{1.5} % 线的密度
\renewcommand{\cftsecdotsep}{1.5} % section引线
\renewcommand{\cftsecleader}{\cftdotfill{\cftsecdotsep}}
\renewcommand{\cftsecpagefont}{}

% 插图清单
\renewcommand{\cftfigpresnum}{\figurename\enspace}

% 表格清单
\renewcommand{\cfttabpresnum}{\tablename\enspace}

% 致谢
\newenvironment{acknowledge}{
\clearpage
\vspace*{-2em}
\phantomsection % 使得hyperref目录能够跳转到正确的位置
\addcontentsline{toc}{section}{致谢} % 添加到目录中
\begin{center}
 \songti \Large \bfseries 致谢\end{center}\vspace{1.1cm}
\setlength{\parindent}{2em}
\setlength{\parskip}{0em}
\setlength{\baselineskip}{22pt} % 22磅行距,首行缩进1字符,段前、段后均为0行间距
\songti
\par
}{
	\par
	\hfill 作者:\studentNameCn

	\hfill \finishedYear\enspace 年\finishedMonth\enspace 月\finishedDay\enspace 日
}

% 附录
\renewenvironment{appendix}{
\clearpage
\vspace*{-2em}
\phantomsection % 使得hyperref目录能够跳转到正确的位置
\addcontentsline{toc}{section}{附录} % 添加到目录中
\begin{center}
 \songti \Large \bfseries 附录\end{center}\vspace{1.1cm}
\setlength{\parindent}{2em}
\setlength{\parskip}{0.5em}
\setlength{\baselineskip}{22pt} % 22磅行距,首行缩进1字符,段前、段后均为0行间距
\songti
\par
}{
}

% 图名称
\renewcommand{\figurename}{图}
\renewcommand{\tablename}{表}
